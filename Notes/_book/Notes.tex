% Options for packages loaded elsewhere
\PassOptionsToPackage{unicode}{hyperref}
\PassOptionsToPackage{hyphens}{url}
%
\documentclass[
]{book}
\usepackage{lmodern}
\usepackage{amssymb,amsmath}
\usepackage{ifxetex,ifluatex}
\ifnum 0\ifxetex 1\fi\ifluatex 1\fi=0 % if pdftex
  \usepackage[T1]{fontenc}
  \usepackage[utf8]{inputenc}
  \usepackage{textcomp} % provide euro and other symbols
\else % if luatex or xetex
  \usepackage{unicode-math}
  \defaultfontfeatures{Scale=MatchLowercase}
  \defaultfontfeatures[\rmfamily]{Ligatures=TeX,Scale=1}
\fi
% Use upquote if available, for straight quotes in verbatim environments
\IfFileExists{upquote.sty}{\usepackage{upquote}}{}
\IfFileExists{microtype.sty}{% use microtype if available
  \usepackage[]{microtype}
  \UseMicrotypeSet[protrusion]{basicmath} % disable protrusion for tt fonts
}{}
\makeatletter
\@ifundefined{KOMAClassName}{% if non-KOMA class
  \IfFileExists{parskip.sty}{%
    \usepackage{parskip}
  }{% else
    \setlength{\parindent}{0pt}
    \setlength{\parskip}{6pt plus 2pt minus 1pt}}
}{% if KOMA class
  \KOMAoptions{parskip=half}}
\makeatother
\usepackage{xcolor}
\IfFileExists{xurl.sty}{\usepackage{xurl}}{} % add URL line breaks if available
\IfFileExists{bookmark.sty}{\usepackage{bookmark}}{\usepackage{hyperref}}
\hypersetup{
  pdftitle={Intermediate R Workshop Notes},
  pdfauthor={Hena R. Ramay},
  hidelinks,
  pdfcreator={LaTeX via pandoc}}
\urlstyle{same} % disable monospaced font for URLs
\usepackage{color}
\usepackage{fancyvrb}
\newcommand{\VerbBar}{|}
\newcommand{\VERB}{\Verb[commandchars=\\\{\}]}
\DefineVerbatimEnvironment{Highlighting}{Verbatim}{commandchars=\\\{\}}
% Add ',fontsize=\small' for more characters per line
\usepackage{framed}
\definecolor{shadecolor}{RGB}{248,248,248}
\newenvironment{Shaded}{\begin{snugshade}}{\end{snugshade}}
\newcommand{\AlertTok}[1]{\textcolor[rgb]{0.94,0.16,0.16}{#1}}
\newcommand{\AnnotationTok}[1]{\textcolor[rgb]{0.56,0.35,0.01}{\textbf{\textit{#1}}}}
\newcommand{\AttributeTok}[1]{\textcolor[rgb]{0.77,0.63,0.00}{#1}}
\newcommand{\BaseNTok}[1]{\textcolor[rgb]{0.00,0.00,0.81}{#1}}
\newcommand{\BuiltInTok}[1]{#1}
\newcommand{\CharTok}[1]{\textcolor[rgb]{0.31,0.60,0.02}{#1}}
\newcommand{\CommentTok}[1]{\textcolor[rgb]{0.56,0.35,0.01}{\textit{#1}}}
\newcommand{\CommentVarTok}[1]{\textcolor[rgb]{0.56,0.35,0.01}{\textbf{\textit{#1}}}}
\newcommand{\ConstantTok}[1]{\textcolor[rgb]{0.00,0.00,0.00}{#1}}
\newcommand{\ControlFlowTok}[1]{\textcolor[rgb]{0.13,0.29,0.53}{\textbf{#1}}}
\newcommand{\DataTypeTok}[1]{\textcolor[rgb]{0.13,0.29,0.53}{#1}}
\newcommand{\DecValTok}[1]{\textcolor[rgb]{0.00,0.00,0.81}{#1}}
\newcommand{\DocumentationTok}[1]{\textcolor[rgb]{0.56,0.35,0.01}{\textbf{\textit{#1}}}}
\newcommand{\ErrorTok}[1]{\textcolor[rgb]{0.64,0.00,0.00}{\textbf{#1}}}
\newcommand{\ExtensionTok}[1]{#1}
\newcommand{\FloatTok}[1]{\textcolor[rgb]{0.00,0.00,0.81}{#1}}
\newcommand{\FunctionTok}[1]{\textcolor[rgb]{0.00,0.00,0.00}{#1}}
\newcommand{\ImportTok}[1]{#1}
\newcommand{\InformationTok}[1]{\textcolor[rgb]{0.56,0.35,0.01}{\textbf{\textit{#1}}}}
\newcommand{\KeywordTok}[1]{\textcolor[rgb]{0.13,0.29,0.53}{\textbf{#1}}}
\newcommand{\NormalTok}[1]{#1}
\newcommand{\OperatorTok}[1]{\textcolor[rgb]{0.81,0.36,0.00}{\textbf{#1}}}
\newcommand{\OtherTok}[1]{\textcolor[rgb]{0.56,0.35,0.01}{#1}}
\newcommand{\PreprocessorTok}[1]{\textcolor[rgb]{0.56,0.35,0.01}{\textit{#1}}}
\newcommand{\RegionMarkerTok}[1]{#1}
\newcommand{\SpecialCharTok}[1]{\textcolor[rgb]{0.00,0.00,0.00}{#1}}
\newcommand{\SpecialStringTok}[1]{\textcolor[rgb]{0.31,0.60,0.02}{#1}}
\newcommand{\StringTok}[1]{\textcolor[rgb]{0.31,0.60,0.02}{#1}}
\newcommand{\VariableTok}[1]{\textcolor[rgb]{0.00,0.00,0.00}{#1}}
\newcommand{\VerbatimStringTok}[1]{\textcolor[rgb]{0.31,0.60,0.02}{#1}}
\newcommand{\WarningTok}[1]{\textcolor[rgb]{0.56,0.35,0.01}{\textbf{\textit{#1}}}}
\usepackage{longtable,booktabs}
% Correct order of tables after \paragraph or \subparagraph
\usepackage{etoolbox}
\makeatletter
\patchcmd\longtable{\par}{\if@noskipsec\mbox{}\fi\par}{}{}
\makeatother
% Allow footnotes in longtable head/foot
\IfFileExists{footnotehyper.sty}{\usepackage{footnotehyper}}{\usepackage{footnote}}
\makesavenoteenv{longtable}
\usepackage{graphicx,grffile}
\makeatletter
\def\maxwidth{\ifdim\Gin@nat@width>\linewidth\linewidth\else\Gin@nat@width\fi}
\def\maxheight{\ifdim\Gin@nat@height>\textheight\textheight\else\Gin@nat@height\fi}
\makeatother
% Scale images if necessary, so that they will not overflow the page
% margins by default, and it is still possible to overwrite the defaults
% using explicit options in \includegraphics[width, height, ...]{}
\setkeys{Gin}{width=\maxwidth,height=\maxheight,keepaspectratio}
% Set default figure placement to htbp
\makeatletter
\def\fps@figure{htbp}
\makeatother
\setlength{\emergencystretch}{3em} % prevent overfull lines
\providecommand{\tightlist}{%
  \setlength{\itemsep}{0pt}\setlength{\parskip}{0pt}}
\setcounter{secnumdepth}{5}
\usepackage{booktabs}
\usepackage[]{natbib}
\bibliographystyle{apalike}

\title{Intermediate R Workshop Notes}
\author{Hena R. Ramay}
\date{2020-05-21}

\begin{document}
\maketitle

{
\setcounter{tocdepth}{1}
\tableofcontents
}
\hypertarget{introduction}{%
\chapter{Introduction}\label{introduction}}

This notebook is created so that the particpants of the workshop can follow along with the instructors. We will constantly update this notebook during the workshop to include answers to the questions asked during the workshop. So Please ask a lot of questions!!

\hypertarget{workshop-schedule}{%
\section{Workshop Schedule}\label{workshop-schedule}}

We will try to cover the following material in the course:

\hypertarget{day-1}{%
\subsection*{Day 1}\label{day-1}}
\addcontentsline{toc}{subsection}{Day 1}

\begin{longtable}[]{@{}ll@{}}
\toprule
Time & Topic\tabularnewline
\midrule
\endhead
9:00 & Introductions and workshop overview\tabularnewline
9:30 & Conditionals and Loops\tabularnewline
10:45 & Functions\tabularnewline
13:00 & Apply family\tabularnewline
14:00 & Advanced plotting ggplot2\tabularnewline
\bottomrule
\end{longtable}

\hypertarget{day-2}{%
\subsection*{Day 2}\label{day-2}}
\addcontentsline{toc}{subsection}{Day 2}

\begin{longtable}[]{@{}ll@{}}
\toprule
Time & Topic\tabularnewline
\midrule
\endhead
9:00 & R markdown basics\tabularnewline
10:45 & Knitr chunks\tabularnewline
13:00 & Rmarkdown themes, adding html and custom css styling\tabularnewline
14:00 & A short project to bring it all together!\tabularnewline
\bottomrule
\end{longtable}

\hypertarget{important-links}{%
\section{Important links}\label{important-links}}

Overview and setup

Lets get started!

\hypertarget{setting-up-a-project}{%
\section{Setting up a project}\label{setting-up-a-project}}

\hypertarget{learning-objectives}{%
\section{Learning Objectives}\label{learning-objectives}}

\begin{itemize}
\tightlist
\item
  Create directories and files
\item
  Understand the difference between absolute and relative paths
\end{itemize}

Before we start using R, we will first review the basics of the Unix shell by setting up our project. Open the terminal in OS X or Linux, or Git Bash in Windows.

Our project will explore the citations and alternative metrics (altmetrics) for articles published in the PLOS family of journals between 2003 and 2010. The data set was compiled by Priem et al.~2012 (\href{http://arxiv.org/abs/1203.4745}{publication},~\href{https://github.com/jasonpriem/plos_altmetrics_study}{code}).

First create a new directory to store the project files called~\texttt{altmetrics}~and then change to that directory.

\begin{verbatim}
mkdir altmetrics
cd altmetrics
\end{verbatim}

Repeat this process to create a subdirectory to store the data files.

\begin{verbatim}
mkdir data
cd data
\end{verbatim}

\hypertarget{finding-your-location}{%
\section{Finding your location}\label{finding-your-location}}

Recall that you can always determine where you are by running the command~\texttt{pwd}, which stands for ``print working directory''. Also, if you run~\texttt{cd}~with no arguments, it takes you to your home directory.

Download the two data files using the links below. Save them to the data subdirectory.

\begin{itemize}
\tightlist
\item
  \href{https://raw.githubusercontent.com/jdblischak/r-intermediate-altmetrics/gh-pages/data/counts-raw.txt.gz}{counts-raw}
\item
  \href{https://raw.githubusercontent.com/jdblischak/r-intermediate-altmetrics/gh-pages/data/counts-norm.txt.gz}{counts-norm}
\end{itemize}

\hypertarget{downloading-files-from-the-command-line}{%
\section{Downloading files from the command line}\label{downloading-files-from-the-command-line}}

When possible, it is best to download files from the command line because that makes the analysis more reproducible. Different systems have different tools installed, which is why we manually downloaded the files for this lesson. Some options include~\texttt{wget},~\texttt{curl}, and~\texttt{rsync}. As an example, here is how to perform the download using~\texttt{wget}.

\begin{verbatim}
wget https://raw.githubusercontent.com/jdblischak/r-intermediate-altmetrics/gh-pages/data/counts-raw.txt.gz
wget https://raw.githubusercontent.com/jdblischak/r-intermediate-altmetrics/gh-pages/data/counts-norm.txt.gz
\end{verbatim}

If~\texttt{wget}~isn't installed on your machine, try~\texttt{curl}:

\begin{verbatim}
curl -O "https://raw.githubusercontent.com/jdblischak/r-intermediate-altmetrics/gh-pages/data/counts-raw.txt.gz"
curl -O  "https://raw.githubusercontent.com/jdblischak/r-intermediate-altmetrics/gh-pages/data/counts-norm.txt.gz"
\end{verbatim}

The first file,~\texttt{counts-raw.txt.gz}, contains the raw counts for each of the articles across all the metrics. The second file,~\texttt{counts-norm.txt.gz}, contains the counts for each of the articles across all the metrics after they have been normalized across disciplines and years.

Confirm that the files downloaded by listing the files in~\texttt{data}.

\begin{verbatim}
ls
\end{verbatim}

\begin{verbatim}
counts-norm.txt.gz counts-raw.txt.gz
\end{verbatim}

Now move back up a directory to~\texttt{altmetrics}. One option would be to specify the absolute path to this directory, e.g.~\texttt{\textasciitilde{}/altmetrics/}~if you created the directory in your home folder. However, an easier option is to use a relative path, which is dependent on the directory you are currently in. The shortcut to move to the directory above is two periods:~\texttt{..}.

\begin{verbatim}
cd ..
\end{verbatim}

Now you should be in the~\texttt{altmetrics}~directory. If you ran the above command again, you would be moved to the directory where you created~\texttt{altmetrics}.

To create files, we'll use the simple text editor~\texttt{nano}. As an argument, you provide the name of an existing file to edit or the name of a new file to create. If you call~\texttt{nano}~without specifying a filename, it will prompt you for a filename when saving. Create a file to practice.

\begin{verbatim}
nano example-file
\end{verbatim}

The commands are listed at the bottom of the screen. The~\texttt{\^{}}~stands for~\texttt{Ctrl}, thus to save type~\texttt{Ctrl-W}~and to exit type~\texttt{Ctrl-X}.

Remove the file with~\texttt{rm}.

\begin{verbatim}
rm example-file
\end{verbatim}

\hypertarget{organizing-a-larger-project}{%
\section{Organizing a larger project}\label{organizing-a-larger-project}}

Projects can grow quickly. For ideas on organizing your own projects, see~\href{http://journals.plos.org/ploscompbiol/article?id=10.1371/journal.pcbi.1000424}{A Quick Guide to Organizing Computational Biology Projects}~by William Noble.

\hypertarget{create-a-readme-file}{%
\section{Create a README file}\label{create-a-readme-file}}

It is a convention to have a file named~\texttt{README}~in a project directory to explain what it contains (both for others and your future self). Use~\texttt{nano}~to create a README file. Include the date and explain that this directory was created for a Software Carpentry workshop.

\href{http://software-carpentry.org/}{Software Carpentry}~\href{https://github.com/jdblischak/r-intermediate-altmetrics}{Source}~\href{mailto:admin@software-carpentry.org}{Contact}~\href{https://jdblischak.github.io/r-intermediate-altmetrics/LICENSE.html}{License}

\hypertarget{conditionals-and-loops}{%
\chapter{Conditionals and Loops}\label{conditionals-and-loops}}

\begin{center}\rule{0.5\linewidth}{0.5pt}\end{center}

\hypertarget{learning-objectives-1}{%
\section{Learning Objectives}\label{learning-objectives-1}}

\begin{itemize}
\item
  Write an if\ldots else statement.
\item ~
  \hypertarget{write-a-for-loop-to-modify-a-data-frame.}{%
  \section{Write a for loop to modify a data frame.}\label{write-a-for-loop-to-modify-a-data-frame.}}
\end{itemize}

\hypertarget{conditional-statements}{%
\section{Conditional statements}\label{conditional-statements}}

Decision making is an important part of programming. This can be achieved in R programming using conditional statements such as \texttt{if} and \texttt{if...else}.

\hypertarget{if}{%
\section{if}\label{if}}

The syntax of an if statement is:

\begin{Shaded}
\begin{Highlighting}[]
\ControlFlowTok{if}\NormalTok{ (test_expression) \{}
\NormalTok{  do_this}
\NormalTok{\}}
\end{Highlighting}
\end{Shaded}

\begin{Shaded}
\begin{Highlighting}[]
\NormalTok{x <-}\StringTok{ }\DecValTok{5}
\ControlFlowTok{if}\NormalTok{ (x }\OperatorTok{>}\StringTok{ }\DecValTok{0}\NormalTok{) \{}
  \KeywordTok{print}\NormalTok{(}\StringTok{"positive number"}\NormalTok{)}
\NormalTok{\}}
\end{Highlighting}
\end{Shaded}

\begin{verbatim}
## [1] "positive number"
\end{verbatim}

\hypertarget{ifelse}{%
\section{if\ldots else}\label{ifelse}}

The syntax of an if\ldots else statement is:

\begin{Shaded}
\begin{Highlighting}[]
\ControlFlowTok{if}\NormalTok{ (test_expression) \{}
\NormalTok{  do_this}
\NormalTok{\} }\ControlFlowTok{else}\NormalTok{ \{}
\NormalTok{    do_that}
\NormalTok{\}}
\end{Highlighting}
\end{Shaded}

The else part is optional and is only evaluated if test\_expression is FALSE. It is important that the \texttt{else} word be in the same line as the closing brace of the \texttt{if} statement.

\begin{Shaded}
\begin{Highlighting}[]
\NormalTok{x <-}\StringTok{ }\DecValTok{1}
\ControlFlowTok{if}\NormalTok{ (x }\OperatorTok{>}\StringTok{ }\DecValTok{0}\NormalTok{)\{}
  \KeywordTok{print}\NormalTok{(}\StringTok{"positive number"}\NormalTok{)}
\NormalTok{\} }\ControlFlowTok{else}\NormalTok{ \{}
    \KeywordTok{print}\NormalTok{(}\StringTok{"negative number"}\NormalTok{)}
\NormalTok{\}}
\end{Highlighting}
\end{Shaded}

\begin{verbatim}
## [1] "positive number"
\end{verbatim}

\hypertarget{nested-ifelse-statements}{%
\section{Nested if\ldots else statements}\label{nested-ifelse-statements}}

You can have more than two test expressions:

\begin{Shaded}
\begin{Highlighting}[]
\ControlFlowTok{if}\NormalTok{ (test_expression1) \{}
\NormalTok{  statement1}
\NormalTok{\} }\ControlFlowTok{else} \ControlFlowTok{if}\NormalTok{ (test_expression2) \{}
\NormalTok{    statement2}
\NormalTok{\} }\ControlFlowTok{else}\NormalTok{ \{}
\NormalTok{      statement4}
\NormalTok{\}}
\end{Highlighting}
\end{Shaded}

\begin{Shaded}
\begin{Highlighting}[]
\NormalTok{x <-}\StringTok{ }\DecValTok{0}
\ControlFlowTok{if}\NormalTok{ (x }\OperatorTok{<}\StringTok{ }\DecValTok{0}\NormalTok{) \{}
   \KeywordTok{print}\NormalTok{(}\StringTok{"negative number"}\NormalTok{)}
\NormalTok{\} }\ControlFlowTok{else} \ControlFlowTok{if}\NormalTok{ (x }\OperatorTok{>}\StringTok{ }\DecValTok{0}\NormalTok{) \{}
   \KeywordTok{print}\NormalTok{(}\StringTok{"positive number"}\NormalTok{)}
\NormalTok{\} }\ControlFlowTok{else}\NormalTok{ \{}
   \KeywordTok{print}\NormalTok{(}\StringTok{"zero"}\NormalTok{)}
\NormalTok{\}}
\end{Highlighting}
\end{Shaded}

\begin{verbatim}
## [1] "zero"
\end{verbatim}

\hypertarget{exercise}{%
\subsection{EXERCISE}\label{exercise}}

Write a simple if\ldots else statement to check if 5 is an odd number and if it is print ``I am odd'', otherwise print ``I am even''.

\hypertarget{loops}{%
\section{Loops}\label{loops}}

Conceptually, a loop is a way to repeat a sequence of instructions under certain conditions. They allow you to automate parts of your code that are in need of repetition.

The easiest and most frequently used loop in R is a for loop. Here is a demonstration of using loops.

\begin{Shaded}
\begin{Highlighting}[]
\NormalTok{year <-}\StringTok{ }\KeywordTok{c}\NormalTok{(}\DecValTok{2015}\NormalTok{,}\DecValTok{2016}\NormalTok{,}\DecValTok{2017}\NormalTok{,}\DecValTok{2018}\NormalTok{)}
\ControlFlowTok{for}\NormalTok{(i }\ControlFlowTok{in} \DecValTok{1}\OperatorTok{:}\KeywordTok{length}\NormalTok{(year)) \{}
  \KeywordTok{print}\NormalTok{(year[i])}
\NormalTok{\}}
\end{Highlighting}
\end{Shaded}

\begin{verbatim}
## [1] 2015
## [1] 2016
## [1] 2017
## [1] 2018
\end{verbatim}

\begin{Shaded}
\begin{Highlighting}[]
\ControlFlowTok{for}\NormalTok{(i }\ControlFlowTok{in} \DecValTok{1}\OperatorTok{:}\KeywordTok{length}\NormalTok{(year)) \{}
  \KeywordTok{print}\NormalTok{(}\KeywordTok{paste0}\NormalTok{(}\StringTok{"the year is "}\NormalTok{,year[i]))}
\NormalTok{\}}
\end{Highlighting}
\end{Shaded}

\begin{verbatim}
## [1] "the year is 2015"
## [1] "the year is 2016"
## [1] "the year is 2017"
## [1] "the year is 2018"
\end{verbatim}

\hypertarget{exercise-1}{%
\subsection{EXERCISE}\label{exercise-1}}

\textbf{Challenging.} Using the \emph{E. coli} metadata from the last section, write a \texttt{for} loop containing an \texttt{if...else} statement to change the genome size into categorical variable with two levels: large and small.

Do this by translating the following sentence into R code: for every element in the genome\_size variable, if the genome size is greater than the mean, change it to ``large'', otherwise, change it to ``small''.

\hypertarget{functions}{%
\chapter{Functions}\label{functions}}

``Intelligence is the ability to avoid doing work, yet getting the work done.''
― Linus Torvalds

\hypertarget{learning-objectives-2}{%
\section{Learning Objectives}\label{learning-objectives-2}}

\begin{itemize}
\tightlist
\item
  Functions have two parts: arguments and body
\item
  Functions have their own environment
\item
  Functions help you repeat code chunks
\end{itemize}

In the last lesson we learnt to write loops to avoid writing the same code multiple time. Now lets learn how to make our code even more efficient and re-use the same code multiple times without copy-paste!

\hypertarget{the-parts-of-a-function}{%
\subsection{The parts of a function}\label{the-parts-of-a-function}}

We've already been using built-in R functions:~\texttt{read.csv},~\texttt{mean},~\texttt{length}, etc. These functions allow us to run the same routine with different inputs.

Let's explore~\texttt{read.csv}~further. All functions in R have two parts: the input arguments and the body. We can see the arguments of a function with the~\texttt{args}.

\begin{verbatim}
args(read.csv)
\end{verbatim}

\begin{verbatim}
function (file, header = TRUE, sep = ",", quote = "\"", dec = ".", 
    fill = TRUE, comment.char = "", ...) 
NULL
\end{verbatim}

So when we pass a character vector like~\texttt{"data/counts-raw."}, this gets assigned to the argument~\texttt{file}. All the other arguments have defaults set, so we do not need to assign them a value.

After the arguments have been assigned values, then the body of the function is executed. We can view the body of a function with~\texttt{body}.

\begin{verbatim}
body(read.csv)
\end{verbatim}

\begin{verbatim}
read.table(file = file, header = header, sep = sep, quote = quote,
    dec = dec, fill = fill, comment.char = comment.char, ...)
\end{verbatim}

\texttt{read.csv}~is very short. It just calls another function,~\texttt{read.table}, using the input file and the default arguments as the arguments passed to~\texttt{read.table}.

When we define our own functions, we use the syntax below. We list the arguments, separated by commas, within the parentheses. The body follows, contained within curly brackets~\texttt{\{\}}.

\begin{verbatim}
function_name <- function(args) {
  body
}
\end{verbatim}

\hypertarget{the-principle-of-encapsulation}{%
\subsection{The principle of encapsulation}\label{the-principle-of-encapsulation}}

An important feature of functions is the principle of encapsulation: the environment inside the function is distinct from the environment outside the function. In other words, variables defined inside a function are separate from variables defined outside the function.

Here's an small example to demonstrate this idea. The function~\texttt{ex\_fun}~takes two input arguments,~\texttt{x}~and~\texttt{y}. It calculates~\texttt{z}~and returns its value.

\begin{verbatim}
ex_fun <- function(x, y) {
  z <- x - y
  return(z)
}
\end{verbatim}

When we run~\texttt{ex\_fun}, the only thing returned to the global environment is the value that was assigned to~\texttt{z}. The variable~\texttt{z}~itself was only defined in the function environment, and does not exist in the global environment.

\begin{verbatim}
ex_fun(3, 10)
\end{verbatim}

\begin{verbatim}
[1] -7
\end{verbatim}

\begin{verbatim}
z
\end{verbatim}

\begin{verbatim}
Error in eval(expr, envir, enclos): object 'z' not found
\end{verbatim}

\hypertarget{environments-are-complicated}{%
\section{Environments are complicated}\label{environments-are-complicated}}

The situation presented above is a simplified version of environments which will serve you well if you treat functions as truly encapsulated. In reality, things are more complicated. For example, if inside a function you have a variable that has not been defined in the function, it will actually search the global environment for this variable. To learn the advanced details, see the chapter~\href{http://adv-r.had.co.nz/Environments.html}{Environments}~in Advanced R by Hadley Wickham.

\hypertarget{the-return-statement}{%
\section{The return statement}\label{the-return-statement}}

R provides the shortcut of not needing to use~\texttt{return}~at the end of the function. Instead, the variable on the last line of the body of the function is returned. This is useful for writing very small functions, but in these lessons we will use~\texttt{return}~to be more explicit about what is happening.

\hypertarget{challenges}{%
\subsection{Challenges}\label{challenges}}

\hypertarget{write-your-own-function}{%
\section{Write your own function}\label{write-your-own-function}}

Write your own function to calculate the mean called~\texttt{my\_mean}. It should take one input argument,~\texttt{x}, which is a numeric vector. Compare your results with the results from R's function~\texttt{mean}. Do you receive the same answer?

\hypertarget{apply-functions-with-purrr-package}{%
\chapter{Apply Functions with purrr package}\label{apply-functions-with-purrr-package}}

\url{https://towardsdatascience.com/functional-programming-in-r-with-purrr-469e597d0229}

It's easier to see the intent of your code, because your eyes are drawn to what's different, not what stays the same.

It's easier to respond to changes in requirements. As your needs change, you only need to make changes in one place, rather than remembering to change every place that you copied-and-pasted the code.

You're likely to have fewer bugs because each line of code is used in more places.

\hypertarget{advanced-plotting}{%
\chapter{Advanced Plotting}\label{advanced-plotting}}

``The greatest value of a picture is when it forces us to notice what we never expected to see''
-John Tukey

\hypertarget{setup}{%
\section{Setup}\label{setup}}

\textbf{1. Install the \texttt{tidyverse} package.}

\begin{Shaded}
\begin{Highlighting}[]
\KeywordTok{library}\NormalTok{(tidyverse)}
\end{Highlighting}
\end{Shaded}

\textbf{2. Download datasets.}

We will be using a dataset containing citation and alternative metrics for articles published in the PLOS family of journals between 2003 and 2010. The data set was compiled by Priem et al 2012 (\href{http://arxiv.org/abs/1203.4745}{publication}).

Download the data onto your computer from \href{https://www.dropbox.com/s/up38z3fd4llanrb/counts-raw.txt?dl=0}{this} dropbox link and move it into a directory on your computer that makes sense.

\textbf{3. Read in data into R.}

\begin{Shaded}
\begin{Highlighting}[]
\NormalTok{counts <-}\StringTok{ }\KeywordTok{read.delim}\NormalTok{(}\StringTok{"data/counts-raw.txt"}\NormalTok{)}

\NormalTok{research <-}\StringTok{ }\KeywordTok{filter}\NormalTok{(counts, articleType }\OperatorTok{==}\StringTok{ "Research Article"}\NormalTok{)}
\end{Highlighting}
\end{Shaded}

\hypertarget{review-of-ggplot2-basics}{%
\section{Review of ggplot2 basics}\label{review-of-ggplot2-basics}}

\texttt{ggplot2} is a plotting package that makes it simple to create complex plots from data in a data frame. Graphics are built step by step by adding new elements. Adding layers in this fashion allows for extensive flexibility and customization of plots.

A plot can be divided into different fundamental parts:

\textbf{Plot = data + aesthetics + geom}

Required building blocks:

\begin{itemize}
\tightlist
\item
  data
\item
  aesthetics - describe how data are mapped to colour, size, shape, location
\item
  geoms - geometric objects like points, lines, shapes
\end{itemize}

Optional building blocks:

\begin{itemize}
\tightlist
\item
  facets - describes how panel plots should be constructed
\item
  stats - statistical transformations like binning, quantiles, smoothing
\item
  coordinates - describes the system in which the locations of the geoms will be drawn
\item
  scales - what scale an aesthetic map uses (ex. male = red, female = blue)
\end{itemize}

To build a ggplot, we will use the following basic template that can be used for different types of plots:

\begin{Shaded}
\begin{Highlighting}[]
\KeywordTok{ggplot}\NormalTok{(}\DataTypeTok{data =} \OperatorTok{<}\NormalTok{DATA}\OperatorTok{>}\NormalTok{, }\DataTypeTok{mapping =} \KeywordTok{aes}\NormalTok{(}\OperatorTok{<}\NormalTok{MAPPINGS}\OperatorTok{>}\NormalTok{)) }\OperatorTok{+}\StringTok{  }\ErrorTok{<}\NormalTok{GEOM_FUNCTION}\OperatorTok{>}\NormalTok{()}
\end{Highlighting}
\end{Shaded}

\begin{enumerate}
\def\labelenumi{\arabic{enumi}.}
\item
  Specify which data set to use for the plot using the \texttt{data} argument.
\item
  Define a ``mapping'' (using the aesthetic (\texttt{aes}) function), by selecting the variables to be plotted and specifying how to present them in the graph, e.g.~as x/y positions or characteristics such as size, shape, color, etc.
\item
  Add ``geoms'' -- graphical representations of the data in the plot (points, lines, bars). \texttt{ggplot2} offers many different geoms; common ones include:
\end{enumerate}

\begin{itemize}
\tightlist
\item
  \texttt{geom\_point()} for scatter plots, dot plots, etc.
\item
  \texttt{geom\_boxplot()} for boxplots.
\item
  \texttt{geom\_histogram()} for histograms.
\item
  \texttt{geom\_barplot()} for barplots.
\item
  \texttt{geom\_line()} for trend lines, time series, etc.
\end{itemize}

\begin{Shaded}
\begin{Highlighting}[]
\NormalTok{p <-}\StringTok{ }\KeywordTok{ggplot}\NormalTok{(research, }\KeywordTok{aes}\NormalTok{( }\DataTypeTok{x =}\NormalTok{ pdfDownloadsCount, }\DataTypeTok{y =}\NormalTok{ wosCountThru2011)) }\OperatorTok{+}\StringTok{ }\KeywordTok{geom_point}\NormalTok{()}
\NormalTok{p}
\end{Highlighting}
\end{Shaded}

\includegraphics{Notes_files/figure-latex/unnamed-chunk-11-1.pdf}

\hypertarget{scales}{%
\section{Scales}\label{scales}}

\hypertarget{faceting}{%
\section{Faceting}\label{faceting}}

\hypertarget{themes}{%
\section{Themes}\label{themes}}

In addition to \texttt{theme\_bw()}, which changes the plot background to white, \texttt{ggplot2} comes with several other themes which can be useful to quickly change the look of your visualization. The complete list of themes is available at \url{https://ggplot2.tidyverse.org/reference/ggtheme.html}. \texttt{theme\_minimal()} and \texttt{theme\_light()} are popular, and \texttt{theme\_void()} can be useful as a starting point to create a new hand-crafted theme.

The \texttt{ggthemes} package provides a wide variety of options (including an Excel 2003 theme). The \texttt{ggplot2} \href{https://www.ggplot2-exts.org}{extensions website} provides a list of packages that extend the capabilities of \texttt{ggplot2}, including additional themes.

\hypertarget{color-palettes}{%
\section{Color palettes}\label{color-palettes}}

You can create your own color palettes using the \texttt{colorRamp()} or \texttt{colorRampPalette()} function.

\texttt{colorRamp()} returns a function that takes values between 0 and 1, indicating the extremes of the color palette.

\texttt{colorRampPalette()} returns a function that takes integer arguments and returns a vector of colours.

\begin{Shaded}
\begin{Highlighting}[]
\NormalTok{cols <-}\StringTok{ }\KeywordTok{colorRamp}\NormalTok{(}\KeywordTok{c}\NormalTok{(}\StringTok{"red"}\NormalTok{, }\StringTok{"blue"}\NormalTok{))}
\KeywordTok{cols}\NormalTok{(}\DecValTok{0}\NormalTok{)}
\end{Highlighting}
\end{Shaded}

\begin{verbatim}
##      [,1] [,2] [,3]
## [1,]  255    0    0
\end{verbatim}

\begin{Shaded}
\begin{Highlighting}[]
\KeywordTok{cols}\NormalTok{(}\FloatTok{0.5}\NormalTok{)}
\end{Highlighting}
\end{Shaded}

\begin{verbatim}
##       [,1] [,2]  [,3]
## [1,] 127.5    0 127.5
\end{verbatim}

\begin{Shaded}
\begin{Highlighting}[]
\KeywordTok{cols}\NormalTok{(}\DecValTok{1}\NormalTok{)}
\end{Highlighting}
\end{Shaded}

\begin{verbatim}
##      [,1] [,2] [,3]
## [1,]    0    0  255
\end{verbatim}

\begin{Shaded}
\begin{Highlighting}[]
\NormalTok{cols <-}\StringTok{ }\KeywordTok{colorRampPalette}\NormalTok{(}\KeywordTok{c}\NormalTok{(}\StringTok{"red"}\NormalTok{, }\StringTok{"blue"}\NormalTok{))}
\KeywordTok{cols}\NormalTok{(}\DecValTok{2}\NormalTok{)}
\end{Highlighting}
\end{Shaded}

\begin{verbatim}
## [1] "#FF0000" "#0000FF"
\end{verbatim}

\begin{Shaded}
\begin{Highlighting}[]
\KeywordTok{cols}\NormalTok{(}\DecValTok{10}\NormalTok{)}
\end{Highlighting}
\end{Shaded}

\begin{verbatim}
##  [1] "#FF0000" "#E2001C" "#C60038" "#AA0055" "#8D0071" "#71008D" "#5500AA"
##  [8] "#3800C6" "#1C00E2" "#0000FF"
\end{verbatim}

Or, you can use the \texttt{RColorBrewer} package to get a premade colour palette.

There are three types of palettes:
* Sequential
* Diverging
* Qualitative

\begin{Shaded}
\begin{Highlighting}[]
\KeywordTok{library}\NormalTok{(RColorBrewer)}
\KeywordTok{display.brewer.all}\NormalTok{()}
\end{Highlighting}
\end{Shaded}

\includegraphics{Notes_files/figure-latex/unnamed-chunk-13-1.pdf}

\hypertarget{multiple-plots}{%
\section{Multiple plots}\label{multiple-plots}}

There are two useful packages for combining multiple plots: \texttt{gridExtra} and \texttt{cowplot}.

\texttt{gridExtra} has two useful functions: \texttt{grid.arrange()} and \texttt{arrangeGrob()}. However, these functions make no attempt at aligning the plot panels; instead, the plots are simply placed into the grid as they are, so the axes are not aligned. If axis alignment is required, the \texttt{plot\_grid()} function of the \texttt{cowplot} package is better. We will try using both here.

\hypertarget{the-gridextra-package}{%
\subsection{\texorpdfstring{The \texttt{gridExtra} package}{The gridExtra package}}\label{the-gridextra-package}}

\hypertarget{the-cowplot-package}{%
\subsection{\texorpdfstring{The \texttt{cowplot} package}{The cowplot package}}\label{the-cowplot-package}}

\hypertarget{saving-plots}{%
\subsection{Saving plots}\label{saving-plots}}

The easiest way to save a plot is using the \texttt{ggsave()} function.

\hypertarget{additional-resources}{%
\section{Additional Resources}\label{additional-resources}}

\url{http://www.sthda.com/english/wiki/be-awesome-in-ggplot2-a-practical-guide-to-be-highly-effective-r-software-and-data-visualization}

\url{http://r-statistics.co/Top50-Ggplot2-Visualizations-MasterList-R-Code.html}

\begin{itemize}
\tightlist
\item
  Mailing list: \url{http://groups.google.com/group/ggplot2}
\item
  Wiki: \url{https://github.com/hadley/ggplot2/wiki}
\item
  Website: \url{http://had.co.nz/ggplot2/}
\item
  StackOverflow: \url{http://stackoverflow.com/questions/tagged/ggplot}
\end{itemize}

Cheatsheet

\hypertarget{rmarkdown}{%
\chapter{Rmarkdown}\label{rmarkdown}}

``If I went back to college again, I'd concentrate on two areas: learning to write and to speak before an audience. Nothing in life is more important than the ability to communicate effectively.''
-- Gerald R. Ford

\hypertarget{writing-in-r-markdown}{%
\section{Writing in R Markdown}\label{writing-in-r-markdown}}

\hypertarget{learning-objectives-3}{%
\section{Learning Objectives}\label{learning-objectives-3}}

Learn how to generate reproducible reports that display your code and results.

When you perform wet lab experiments, what information do you put in your lab notebook? You probably include the protocol you used to run the experiment, information about the samples and reagents used in the protocol, and at the end you'll likely include your results (for instance, a picture of a gel). This essentially creates a report of your experiment.

You can do the same with your dry lab analyses using a tool called R Markdown. Why would we want to do this?

\begin{itemize}
\item
  Your method, results, and interpretation are stored in one place
\item
  If you update your methodology, you can easily update your results with the click of a button, rather than copying and pasting.
\item
  You~\emph{could}~cut and paste your code and results into Word or Power Point, but that will make rerunning your code challenging, as Word often introduces hidden characters.
\end{itemize}

R Markdown is a fairly simple language you can use to generate reports that incorporate bits of R code along with the output they produce. There are two steps to generating reports with R Markdown and RStudio:

\begin{enumerate}
\def\labelenumi{\arabic{enumi}.}
\tightlist
\item
  Write your code in R Markdown.
\item
  Assemble your report as either HTML or a PDF using the package rmarkdown.
\end{enumerate}

Next, let's run through the demo R Markdown file to see some of the options. Go up to~\texttt{File}~-\textgreater~\texttt{New\ File}~-\textgreater~\texttt{R\ Markdown}.

\begin{figure}
\centering
\includegraphics{img/markdown.png}
\caption{Set up new R Markdown file}
\end{figure}

Set up new R Markdown file

A screen will pop up asking us what kind of document we wish to create. Let's name our demo report ``Trial Report'' and fill in your name. Ensure that ``Document'' is highlighted to the left and that ``HTML'' is chosen. Click ``Ok''.

\begin{figure}
\centering
\includegraphics{img/mdname.png}
\caption{Choose HTML}
\end{figure}

Choose HTML

Now we have the example R Markdown file open. The first thing you'll notice at the top is a header which includes your name, the title of the document, the date, and a field called output. This header tells the package rmarkdown some information it might need about your document, including what format you want the final report rendered in.

The next thing you'll notice is white space with some text describing an R Markdown document. White space in this document represents text of the report you would like to display. You can put anything here describing your analysis, results, etc. and it will be recognized as text and not R code. This white space is interpreted as Markdown language, so you can use any of the tricks we learned in the last lesson to make lists, bold certain words, or create headers in your document.

In this trial script, you'll see how some of these markdown elements are used. For example, the word knit is in bolded (using asterisks), and there are code chucks near the bottom that say echo = FALSE.

\begin{figure}
\centering
\includegraphics{img/firstmd.png}
\caption{Demo R Markdown Document}
\end{figure}

Demo R Markdown Document

In addition to the white space, you'll gray blocks that have ``` at the top and bottom. These are called chunks. If the start of a chunk has \{r\} at the end of the ticks, the code will be run and both it and its output will be displayed in the rendered HTML. In your R Markdown, the code will look like:

\begin{verbatim}

```r
summary(cars)
\end{verbatim}

\begin{verbatim}
##      speed           dist       
##  Min.   : 4.0   Min.   :  2.00  
##  1st Qu.:12.0   1st Qu.: 26.00  
##  Median :15.0   Median : 36.00  
##  Mean   :15.4   Mean   : 42.98  
##  3rd Qu.:19.0   3rd Qu.: 56.00  
##  Max.   :25.0   Max.   :120.00
\end{verbatim}

\begin{verbatim}

In your final report, the code will look like:
\end{verbatim}

summary(cars)

\begin{verbatim}


Let's add a new chunk to end this demo document. To do so, either you can enter three backticks in a row, followed by {r}, or you can click on the green `Chunks`button and chose`Insert Chunk`. Additionally, there's a keyboard short cut which is `ctrl`+`alt`+`i`which will also pop up a chunk in an R Markdown document.

![Insert Chunk](https://jdblischak.github.io/r-intermediate-altmetrics/fig/insert-chunk.png)

Insert Chunk

In the chunk, let's just examine the dimensions of the `car`mdataset:
\end{verbatim}

\begin{Shaded}
\begin{Highlighting}[]
\KeywordTok{dim}\NormalTok{(cars)}
\end{Highlighting}
\end{Shaded}

\begin{verbatim}
## [1] 50  2
\end{verbatim}

\begin{verbatim}

You can actually send the code straight from the chunks over to console to be evaluated in two ways. First, you can highlight the code you want to run in the chunk and hit the `Run` button, which is located in the top right corner of the pane. Additionally, you can use the keyboard shortcut `ctrl`+`alt`+`c`. This allows you to iteratively write an test code in RStudio, rather than having to render the full report everytime you add a bit of new code.

These are the basics of writing R Markdown, but we still need to generate a report. To do this, click on the button on the top bar that says "Knit HMTL". This will prompt you to save the file. Go ahead and save this file as `Rmarkdown_demo.Rmd` in the altmetrics directory. The ending of the file `.Rmd` indicates that this is an R Markdown file.

![Knit R Markdown](https://jdblischak.github.io/r-intermediate-altmetrics/fig/knit_rmarkdown.png)

Knit R Markdown

When you click on this link, you see in the console that RStudio is running and rendering your R Markdown file. What is actually happening is RStudio is running the function `render`, which is part of the rmarkdown package. There are two things the command `render` does. First, it converts the R Markdown file to a Markdown file using the command `knit` from the knitr package (hence why rendering is called knitting). The second step is then the Markdown file is converted to the final file format (HTML, PDF, or Word).

The final result is that an HMTL file will pop up where you'll see the report. You can see the header has been rendered, there are code and results chunks displayed, and even plots are shown right in the report.

Also, if you now look in the altmetrics folder, you'll see an HTML file of the name Rmarkdown_demo.html. When `render` is run, it saves the current version of the .Rmd file and the generated HTML file in the directory it is stored in.


Knitr Chunk Options
-------------------

Learning Objectives
-------------------

Learn how format chunks in R Markdown to display only the information you want to display.

You've learned the basics of how to incorporate markdown syntax with code chunks in an R Markdown file. Let's explore some additional options for making your code chunks appear the way you want them to appear in your reports. There are many ways to customize your chunks and you can explore all of the options by examining the [documentation](http://yihui.name/knitr/options/#chunk_options). Here, we'll introduce you to some of the most useful options that you might use frequently.

The first thing you may want to consider is naming your code chunks, which makes degubbing easier, especially if you have a long script. Chunk names must be unique to each chunk.

Write the name of your chunk after the {r}, like: `{r chunk_name}`

R Markdown:
\end{verbatim}

\begin{Shaded}
\begin{Highlighting}[]
\KeywordTok{summary}\NormalTok{(cars)}
\end{Highlighting}
\end{Shaded}

\begin{verbatim}
##      speed           dist       
##  Min.   : 4.0   Min.   :  2.00  
##  1st Qu.:12.0   1st Qu.: 26.00  
##  Median :15.0   Median : 36.00  
##  Mean   :15.4   Mean   : 42.98  
##  3rd Qu.:19.0   3rd Qu.: 56.00  
##  Max.   :25.0   Max.   :120.00
\end{verbatim}

\begin{verbatim}

Rendered:
\end{verbatim}

summary(cars)

\begin{verbatim}



You can use RStudio to navigate to chunks based on their names, which can be especially useful as your script gets long. Click on the bottom left bar where it says `(Top Level)` and you'll see all of the chunk names in your script appear. Additionally, naming your chunks will be beneficial to identify errors in your code or slow sections when knitting your report.

Sometimes you may not want to see the code that produced a particular result in your report. You can have codeblocks in your R Markdown that are evaluated, but the code is not displayed in the final report by including `echo=FALSE` after the `{r chunk_name}`.

R Markdown:
\end{verbatim}

\begin{verbatim}
##      speed           dist       
##  Min.   : 4.0   Min.   :  2.00  
##  1st Qu.:12.0   1st Qu.: 26.00  
##  Median :15.0   Median : 36.00  
##  Mean   :15.4   Mean   : 42.98  
##  3rd Qu.:19.0   3rd Qu.: 56.00  
##  Max.   :25.0   Max.   :120.00
\end{verbatim}

\begin{verbatim}

Rendered:
\end{verbatim}

summary(cars)

\begin{verbatim}

Conversely, sometimes you may want to see the code, but not the output once the code is evalutated. To do so, you can include `results="hide"` after the chunk_name:

R Markdown:
\end{verbatim}

\begin{Shaded}
\begin{Highlighting}[]
\KeywordTok{summary}\NormalTok{(cars)}
\end{Highlighting}
\end{Shaded}

\begin{verbatim}

Rendered:
\end{verbatim}

summary(cars)

\begin{verbatim}

Sometimes, you may want to write a report where both the code and the output are suppressed. Why would you want to do that? Perhaps you're sending a report to a collaborator and you only want them to see the final figures, but not any data manipulation steps in the middle. To include a chunk that is evaluated, but no output is displayed - neither the code nor the results - put `include=FALSE` after the chunk_name:

R Markdown:
\end{verbatim}

\begin{verbatim}

Rendered (I swear there's a chunk after this! It's just invisible!):

There are tons of other options you can include in your chunks: sizing your figures and whether or not to display error or warning messages. As you write reports of your own data analysis, you can look up these options to create a report formatted in the way you want.

In addition to code chunks, you may want to include the results of an evaluation in line with regular text. For instance, you may want to describe the data in a paragraph, and include the number of individuals in that paragraph. To do so, you can indicate that a code box should be evaluated as R by including a lowercase r.

R Markdown:
\end{verbatim}

\begin{verbatim}
The _cars_ dataset included in this analysis contains records for 50 cars.
\end{verbatim}

\begin{verbatim}

Rendered:
\end{verbatim}

The \emph{cars} dataset included in this analysis contains records for 50 cars.
```

\hypertarget{project}{%
\chapter{Project}\label{project}}

2 hours project in which we make plots, write functions and generate a nice html report

\hypertarget{interesting-packages}{%
\chapter{Interesting Packages}\label{interesting-packages}}

\begin{enumerate}
\def\labelenumi{\arabic{enumi}.}
\item
  Cowplot
\item
  forcats
\item
  biobroom
\item
\end{enumerate}

\hypertarget{faqs}{%
\chapter{FAQs}\label{faqs}}

A subset of questions that were asked during the session

  \bibliography{book.bib,packages.bib}

\end{document}
